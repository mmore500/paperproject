\begin{figure}[h]
\begin{minipage}{0.6\textwidth}
  \includegraphics[width=\linewidth]{binder/binder-wse-traits.ipynb/binder/teeplots/wse-traits/errorbar=ci+hue=initial-conditions+kind=line+palette=set2-r+row=population-size+style=initial-conditions+viz=relplot+what=counter-based+x=available-beneficial-mutations+y=fix-prob+ext=.pdf}%
\end{minipage}%
\begin{minipage}{0.4\textwidth}
  \caption{
  \textbf{Initial supply of hypermutators influences normomutator resilience to available adaptive potential.}
  \footnotesize
  Lineplot strips track hypermutator fixation probability across surveyed levels of adaptive potential, with error bands providing bootstrapped 95\% confidence intervals.
  Simulations were conducted on WSE, using a population size of 256 agents per PE and counter-based genome model.
  Results using a site-explicit model were similar, provided in Supplementary Figures \cref{fig:denovo-5050-conditions-combined-site-explicit,fig:wse-site-explicit-counter-based}.
  Supplementary Figure \ref{fig:fixheat-wse-256atile} details results in a tabular format.
    }
    \label{fig:denovo-5050-conditions-combined}
  \end{minipage}
\end{figure}
