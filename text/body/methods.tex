\section{Methods} \label{sec:methods}


% drafted by Gemini, edited by MAM 2026-02-09
\subsection*{ABM v.2026-02-09-allele-abm}

\subsubsection*{Notation and Parameters}
\begin{center}
\begin{tabular}{r l l p{6cm}}
    \textbf{Symbol} & \textbf{Variable} & \textbf{Domain} & \textbf{Description} \\ \hline
    $N$ & \texttt{POP\_SIZE} & $\in \mathbb{N}$ & Total count of hosts in the simulation \\
    $L$ & \texttt{N\_SITES} & $\in \mathbb{N}$ & Number of loci in the pathogen genome \\
    $g$ & \texttt{pathogen\_genomes} & $\in \{0,1\}^L$ & Pathogen genotype (bitstring) \\
    $a$ & - & $\in \{0,1\}$ & Specific allele at a given locus \\
    $H$ & \texttt{host\_immunities} & $\in [0,1]^{N \times L \times 2}$ & Host immunity history tensor \\
    $I$ & \texttt{host\_statuses} & $\in \mathbb{N}_0$ & Days since infection (0 is susceptible) \\
    $\alpha$ & \texttt{IMMUNE\_STRENGTH} & $\in [0,1]$ & Immune efficacy per locus \\
    $\beta$ & \texttt{within\_host\_b} & $\in \mathbb{R}^+$ & Within-host selection coefficient \\
    $\tau$ & \texttt{within\_host\_t} & $\in \mathbb{R}^+$ & Effective infection duration for mutation \\
    $\omega$ & \texttt{WANING\_RATE} & $\in [0,1]$ & Daily geometric decay rate \\
    $\mu$ & \texttt{MUTATION\_RATE} & $\in [0,1]$ & Base mutation probability \\
\end{tabular}
\end{center}

\subsubsection*{System State}
The model tracks a population of $N$ well-mixed hosts.
Each host may be infected by one pathogen strain.
To begin the simulation, 100 hosts are infected with a common ancestral strain (wildtype, all zeros).

Each pathogen strain is defined by a binary string $g$ of length $L$.
Host $i$ maintains an immunity history $H_{i}$, where $H_{i,\ell,a}$ quantifies the defense against allele $a$ at locus $\ell$.

\subsubsection*{Transmission Dynamics}
In each time step, every host $i$ selects a single random contact $j$ from the entire population.
Transmission is only possible if host $i$ is susceptible ($I_i=0$) and the contact $j$ is infected ($I_j > 0$).
The susceptibility of host $i$ depends on the specific alleles carried by the contact's genome $g_j$.
We calculate the probability of infection as the product of failure probabilities at each locus $\ell$:
$$
P(\text{infection}) \propto \prod_{\ell=1}^{L} \left( 1 - \alpha \cdot H_{i, \ell, g_{j,\ell}} \right)
$$
Transmission succeeds if a random draw falls below this calculated probability.

\subsubsection*{Mutation Mechanism}
Mutations are driven by the immune pressure difference between the current allele $a$ and the potential mutant allele $1-a$.
The growth advantage $b$ for the mutant at locus $\ell$ is calculated as:
$$
b = 1 + \beta (H_{i, \ell, a} - H_{i, \ell, 1-a})
$$
The probability of a mutation occurring during the infection period is derived from the net growth of the mutant lineage:
$$
P(\text{mutate}) = \frac{\mu}{b - 1} \left( e^{(b - 1)\tau} - 1 \right)
$$
New mutations are applied immediately to the transmitting genome upon infection of a new host.

\subsubsection*{Immunity Updates}
Waning occurs every step via global geometric decay:
$$H(t+1) = H(t)(1 - \omega)$$
Recovery occurs when the infection duration exceeds the inverse recovery rate.
Upon clearing an infection with genome $g$, the host's immunity to the specific alleles present in that genome saturates:
$$
H_{i, \ell, g_\ell} \leftarrow 1.0 \quad \forall \ell \in \{1, \dots, L\}
$$

\subsection{Software and Data Availability} \label{sec:materials}

Simulation code, configuration files, and batch scripts for this work are available via Zenodo at \url{TODO}.
Postprocessing scripts and executable notebooks for this work are hosted separately at \url{TODO}.
Data and supplemental materials are available via the Open Science Framework \url{TODO} \citep{foster2017open}.

This project benefited significantly from open-source scientific software \citep{2020SciPy-NMeth,harris2020array,reback2020pandas,mckinney-proc-scipy-2010,waskom2021seaborn,hunter2007matplotlib,moreno2023teeplot}.
