\section{Results and Discussion} \label{sec:results}

Our research in this work investigates the interactions between population size, adaptive potential, and population structure in influencing selection on hypermutator traits within asexual populations.
In our first experiments, we replicate previous work by \citet{raynes2018sign} showing sign-change effects of population size on hypermutator favorability.
Subsequent experiments investigate how these dynamics change when limited beneficial mutations are available and their sensitivity to initial abundance of hypermutators in the population.
Finally, experiments survey behavior across three regimes of population spatial structure.
We find that the sign effect of population size on hypermutator favorability is dependent on spatial structure.

\subsection{Scaling Population Size}
\label{sec:scaling-population-size}

In a first set of experiments, we tested how hypermutator fixation probability varies with changes in population size.

Previous work by \citet{raynes2018sign} has demonstrated that hypermutator dynamics can transition through several regimes with population scale.
In the first regime, when population size is very small and normomutators/hypermutators start in even proportion, fixation of either is nearly equiprobable, owing to the overwhelming influence of stochastic effects.
Subsequently, in medium-sized populations, hypermutators become disfavored by selection.
Intuitively, this disfavor can be attributed to the power of selection becoming sufficiently sensitive to penalize the mutational load imposed by hypermutator traits.
However, in sufficiently large populations, hypermutator fixation can move towards certainty.
This is because, in asexual populations, hypermutator alleles that discover a significant beneficial mutation can gain in population share by association with that beneficial allele.
Population size increases the probability of at least one hypermutator discovering a beneficial mutation, and so ``hitch-hiking'' to sweep out normomutators.

\begin{figure}[h]
  % adapted from https://tex.stackexchange.com/a/122813/316176
  \captionsetup[subfigure]{justification=raggedright}
  \begin{minipage}{0.7\textwidth}

    \begin{minipage}{0.04\textwidth}~\end{minipage}%
    \begin{minipage}{0.44\textwidth}
      \centering
      \itshape
      \textbf{unlimited} beneficial mutations
    \end{minipage}%
    \begin{minipage}{0.34\textwidth}
      \centering
      \itshape
      \textbf{one} beneficial mutation
    \end{minipage}

    ~\vspace{-1.5ex}

    % Top subfigure
    \begin{subfigure}[b]{\linewidth}
      \begin{minipage}{0.88\textwidth}
        \begin{minipage}{0.53\textwidth}
          \includegraphics[height=2.9cm, trim={0cm 0.2cm 3.9cm 0.8cm}, clip]{binder/binder-wse-5050-spatial2d-32atile-infben-traits.ipynb/binder/teeplots/wse-5050-spatial2d-32atile-infben-traits/errorbar=ci+hue=genotype+num-abm=(inf,)+style=genotype+viz=size-fixation-areaplot+x=population-size+y=fixation-probability+ext=.pdf}%
        \end{minipage}%
        \begin{minipage}{0.47\textwidth}
          \includegraphics[height=2.9cm, trim={1.3cm 0.2cm 3.9cm 0.8cm}, clip]{binder/binder-wse-5050-spatial2d-32atile-infben-traits.ipynb/binder/teeplots/wse-5050-spatial2d-32atile-infben-traits/errorbar=ci+hue=genotype+num-abm=(1.0,)+style=genotype+viz=size-fixation-areaplot+x=population-size+y=fixation-probability+ext=.pdf}
        \end{minipage}
      \end{minipage}%
      \hspace{-3ex}%
      \begin{minipage}{0.12\textwidth}
        \raggedright
        \caption{\footnotesize 32 agents per PE\\~\\~\\}
        \label{fig:wse-inf-one:32}
      \end{minipage}%
    \end{subfigure}%

    % Bottom subfigure - Adjusted layout to match the top
    \begin{subfigure}[b]{\linewidth}
      \begin{minipage}{0.88\textwidth}
        \begin{minipage}{0.53\textwidth}
          \includegraphics[height=2.9cm, trim={0cm 0.2cm 3.9cm 0.8cm}, clip]{binder/binder-wse-5050-spatial2d-2048atile-infben-traits.ipynb/binder/teeplots/wse-5050-spatial2d-2048atile-infben-traits/errorbar=ci+hue=genotype+num-abm=(inf,)+style=genotype+viz=size-fixation-areaplot+x=population-size+y=fixation-probability+ext=.pdf}%
        \end{minipage}%
        \begin{minipage}{0.47\textwidth}
          \includegraphics[height=2.9cm, trim={1.3cm 0.2cm 3.9cm 0.8cm}, clip]{binder/binder-wse-5050-spatial2d-2048atile-infben-traits.ipynb/binder/teeplots/wse-5050-spatial2d-2048atile-infben-traits/errorbar=ci+hue=genotype+num-abm=(1.0,)+style=genotype+viz=size-fixation-areaplot+x=population-size+y=fixation-probability+ext=.pdf}
        \end{minipage}
      \end{minipage}%
      \hspace{-3ex}%
      \begin{minipage}{0.1\textwidth}
        \caption{\footnotesize 2,048 agents per PE\\~\\~\\}
        \label{fig:wse-inf-one:2048}
      \end{minipage}%
    \end{subfigure}%

~\vspace{-1.2ex}

\includegraphics[width=\textwidth, trim={0cm 4.7cm 5.3cm 0cm}, clip]{binder/binder-wse-5050-spatial2d-2048atile-infben-traits.ipynb/binder/teeplots/wse-5050-spatial2d-2048atile-infben-traits/col=available-beneficial-mutations+errorbar=sd+hue=genotype+kind=line+style=genotype+viz=relplot+x=population-size+y=fixation-probability+ext=}

  \end{minipage}%
  \begin{minipage}{0.3\textwidth}
    \caption{%
      \textbf{Restricted adaptive potential favors normomutators in large populations.}
      \footnotesize
      Area plots compare hyper- versus normo-mutator fixation probabilities across surveyed population sizes.
      When adaptive potential is unlimited (left column), hypermutators are strongly favored in large populations, owing to their capacity to more rapidly discover beneficial mutations.
      Right column shows fixation outcomes across population sizes, with adaptive potential restricted to just one beneficial mutation.
      As before, hypermutators gain favor in intermediate population sizes.
      However, under the adaptation-restricted regime, normomutators regain favor at very large popukation sizes.
      Top panel, \ref{fig:wse-inf-one:32}, shows results with subpopulation size of 32 agents per PE, scaling population size up to 24.4 million (TODO check this).
      Bottom panel, \ref{fig:wse-inf-one:2048}, reports 2,048 agents per PE with population sizes up to 1.5 billion.
      Experiments were conducted on WSE with Poisson genome model, using populations initialized with a 50/50 mix of normo- and hyper-mutators.
      Error bands indicate bootstrapped 95\% confidence interval.
    }
    \label{fig:wse-inf-one}
  \end{minipage}
\end{figure}


As shown in the left panel of Figure \ref{fig:wse-inf-one:32}, we also find a sign-change phenomenon with population scale, as previously reported by \citet{raynes2018sign}.
Applying WSE accelerator hardware allowed us to test population sizes orders of magnitude beyond previously reported upper ceilings of tens of thousands of agents.
However, even at billion-agent scales, no qualitative difference emerges between large and very large population sizes with respect to hypermutator fixation probability.
With the largest-tested density of 2,048 agents per deme, shown as the left panel of Figure \ref{fig:wse-inf-one:2048}, coarse spacing between surveyed population sizes obscures the original sign-change phenomenon in small and medium-sized populations.
Nonetheless, hypermutators retain strong selective favor across large and very large population sizes.

\subsection{Restricting Adaptive Potential}
\label{sec:restricting-adaptive-potential}

Prediction of a strong tendency for hypermutator fixation in large populations prompts a converse question: what conditions are necessary for large populations to resist hypermutator invasion?
One assumption of the above result is availability of abundant potential beneficial mutations.
Indeed, agent-based modeling work has largely yet to explore scenarios where few beneficial mutations are available \citep{raynes2013effect,raynes2018sign,raynes2019selection,raynes2019migration}.
Recent empirical work, however, has shown indications that fitness landscape characteristics can influence hypermutator favorability.

To investigate this question, we next performed experiments where the supply of possible beneficial mutations was constrained.
The right panels of Figures \cref{fig:wse-inf-one:32,fig:wse-inf-one:2048} show hypermutator fixation probabilities where agents could accrue no more than one beneficial mutation.
At both 32- and 2,048 agents per deme, and an additional, third regime of selection on hypermutator traits appears past population sizes of 1 million agents.
Beyond this point, normomutators reliably drive hypermutators to extinction.
We expect this effect to be associated with sufficient scale that at least one normomutator reliably discovers the beneficial mutation before the hypermutator trait fixes.
Then, given the large population size, selection should reliably advantage normomutators' reduced deleterious mutation load.

\begin{figure*}

\begin{minipage}{0.65\textwidth}
  \includegraphics[width=\textwidth]{binder/binder-wse-5050-spatial2d-2048atile-traits.ipynb/binder/teeplots/wse-5050-spatial2d-2048atile-traits/col=available-beneficial-mutations+errorbar=ci+hue=genotype+style=genotype+viz=size-fixation-areaplot+x=population-size+y=fixation-probability+ext=.pdf}%
  \end{minipage}
\begin{minipage}{0.3\textwidth}
\caption{
\textbf{Hypermutators regain favor in large populations when several beneficial mutations are available.}
\footnotesize
TODO WSE with 2,048 agents per PE.
Error bands show bootstrapped 95\% CI intervals.
TODO site-explicit model
}
\end{minipage}

\end{figure*}


Next, we tested the sensitivity of normomutator persistence in large populations to the amount of adaptive potential available.
Figure \ref{fig:avail-ben-muts} shows fixation curves where one, two, three, four, and five beneficial mutations are available, conducted with 256 agents per deme.
Under surveyed conditions, with 2D spatial structure and hypermutators beginning in equal proportion to normomutators, resilience to hypermutator fixation is very fragile.
Normomutator persistence rapidly decays with increases in adaptive potential;
even in billion-agent populations, hypermutators begin to regain favor past three beneficial mutations available.

\begin{figure*}

\begin{minipage}{0.65\textwidth}
  \includegraphics[width=\textwidth]{binder/binder-cupy-5050-traits.ipynb/binder/teeplots/cupy-5050-traits/col=available-beneficial-mutations+errorbar=ci+hue=population-structure+kind=line+palette=hsv+style=population-structure+viz=relplot+x=population-size+y=fixation-probability+ext=.pdf}%
\end{minipage}
\begin{minipage}{0.3\textwidth}
\caption{
\textbf{Well-mixed populations experience similar effects of adaptive potential on normalized fixation probability.}
\footnotesize
In both scenarios, as available beneficial mutations are increased, hypermutators gain favor in progressively larger population sizes.
Simulations were conducted on GPU using the counter-based genome model, with populations initialized to a 50/50 mix of normo- and hypermutators.
Subpopulations comprised 256 agents per PE.
Error bands show bootstrapped 95\% confidence intervals.
}
\label{fig:avail-ben-muts-gen}
\end{minipage}

\end{figure*}


\subsection{Effects of Background Hypermutator Prevalence}
\label{sec:background-hypermutator-prevalence}

\begin{figure}[h]
  % adapted from https://tex.stackexchange.com/a/122813/316176
  \captionsetup[subfigure]{justification=raggedright}
  \begin{minipage}{\textwidth}

    \begin{minipage}{0.1\textwidth}~\end{minipage}%
    \begin{minipage}{0.35\textwidth}
      \centering
      \itshape
      {\large
      \textbf{50/50} conditions
      }
    \end{minipage}%
    \begin{minipage}{0.45\textwidth}
      \centering
      \itshape
      {\large
      \textbf{de novo} conditions
      }
    \end{minipage}

    ~\vspace{-0.7ex}

    % Top subfigure
    \begin{subfigure}[b]{\linewidth}
        \begin{minipage}{0.42\textwidth}
          ~
          % TODO
          \includegraphics[height=5.5cm, trim={0cm 0.8cm 0cm 0cm}, clip]{binder/binder-wse-5050-spatial2d-traits.ipynb/binder/teeplots/wse-5050-spatial2d-traits/col=population-size+errorbar=ci+hue=genotype+layout=wide+viz=size-fixation-cliffplot+x=fixation-probability+y=available-beneficial-mutations+ext=.pdf}%
        \end{minipage}%
        \begin{minipage}{0.06\textwidth}
          ~
        \end{minipage}%
        \begin{minipage}{0.38\textwidth}
          ~
          % TODO
          \includegraphics[height=5.5cm, trim={1.2cm 0.8cm 0cm 0cm}, clip]{binder/binder-wse-denovo-spatial2d-traits.ipynb/binder/teeplots/wse-denovo-spatial2d-traits/col=population-size+col-label=+errorbar=ci+hue=genotype+layout=wide+viz=size-fixation-cliffplot+x=fixation-probability+y=available-beneficial-mutations+ext=.pdf}%
        \end{minipage}%
      \begin{minipage}{0.12\textwidth}
        \raggedright
        \large
        \caption{Poisson\\ model}
        \label{fig:denovo-5050-conditions:poisson}
        \vspace{20ex}
      \end{minipage}%
    \end{subfigure}%

    \vspace{-10ex}

    % Bottom subfigure
    \begin{subfigure}[b]{\linewidth}
        \begin{minipage}{0.41\textwidth}
          \includegraphics[height=5.5cm, trim={0cm 0.8cm 0cm 0cm}, clip]{binder/binder-wse-5050-spatial2d-explicitsite-traits.ipynb/binder/teeplots/wse-5050-spatial2d-explicitsite-traits/col=population-size+errorbar=ci+hue=genotype+layout=wide+viz=size-fixation-cliffplot+x=fixation-probability+y=available-beneficial-mutations+ext=.pdf}%
        \end{minipage}%
        \begin{minipage}{0.06\textwidth}
          ~
        \end{minipage}%
        \begin{minipage}{0.38\textwidth}
          \includegraphics[height=5.5cm, trim={1.2cm 0.8cm 0cm 0cm}, clip]{binder/binder-wse-denovo-spatial2d-explicitsite-traits.ipynb/binder/teeplots/wse-denovo-spatial2d-explicitsite-traits/col=population-size+col-label=+errorbar=ci+hue=genotype+layout=wide+viz=size-fixation-cliffplot+x=fixation-probability+y=available-beneficial-mutations+ext=.pdf}%
        \end{minipage}%
      \begin{minipage}{0.12\textwidth}
        \raggedright
        \large
        \vspace{10ex}
        \caption{Site-explicit model}
        \label{fig:denovo-5050-conditions:site-explicit}

        \includegraphics[height=5.5cm, trim={7.3cm 0cm 0cm 0cm}, clip]{binder/binder-cupy-denovo-spatial1d-traits.ipynb/binder/teeplots/cupy-denovo-spatial1d-traits/col=population-size+errorbar=ci+hue=genotype+layout=skinny+viz=size-fixation-cliffplot+x=fixation-probability+y=available-beneficial-mutations+ext=.pdf}%
      \end{minipage}%
    \end{subfigure}%

  \end{minipage}

  \vspace{-10ex}

  \begin{minipage}{\textwidth}
    \caption{%
      \textbf{Initial supply of hypermutators influences normomutator resilience to available adaptive potential.}
      \footnotesize
      Areaplot strips track hypermutator fixation probability across surveyed levels of adaptive potential, with error bands providing bootstrapped 95\% confidence interval.
      Strips are arranged left-to-right with ascending population size, indicated on top axis.
      In left column, populations are initialized with half of agents as hypermutators.
      In right column, populations are initialized with all normomutators; hypermutators must arise \textit{de novo}, with probability $10^{-6}$ per daughter.
      The \textit{de novo} regime boosts normomutator resilience to adaptive potential, with large population sizes reliably resisting hypermutator fixation through upwards of 10 available beneficial mutations.
      Top panel \ref{fig:denovo-5050-conditions:poisson} gives results for Poisson genome model.
      Bottom panel \ref{fig:denovo-5050-conditions:site-explicit} gives results under site-explicit genome model, which are similar.
      Simulations were conducted on WSE, using per-PE population size of 256.
    }
    \label{fig:denovo-5050-conditions}
  \end{minipage}
\end{figure}


Having observed large population size stabilize normomutators against extinction when few opportunities for adaptive mutation exist, we next sought to assess how robustly such a protective effect might manifest in natural populations.
In this regard, we sought to test the role of initial conditions in hypermutator fixation outcomes by relaxing the assumption that hypermutators begin ``50/50'' in equal proportion to normomutators.
Although such a scenario provides a useful design for benchtop experiments \citep{raynes2018sign}, against a background of evolutionary stasis it would, instead, be expected that normomutators instead exist in the population at some low background rate resulting equilibrium between purifying selection and spontaneous hypermutator originations \citep{desai2011balance,johnson1999approach}.

Prior work, applying both mathematical analysis and agent-based experiments, has shown per-capita favorability of hypermutator traits to be density independent \citep{raynes2019selection}.
Thus, we would expect a lower background rate for hypermutator alleles to reduce the net probability of normomutator extinction.

To test this question, we conducted ``\textit{de novo}'' treatments where hypermutators arose spontaneously rather than being explicitly included in the initial population.
Given that we assume the availability of some nonzero number of adaptive mutations, such a scenario might be applicable to cases where an abrupt change in environmental context introduces adaptive potential, such as the sudden introduction of a novel stressor.
Note that, in our simulation experiments, we assumed hypermutators to originate with probability equivalent to beneficial mutations, $10^{-6}$.

Figure \ref{fig:denovo-5050-conditions} compares surveyed hypermutator fixation probabilities with and without hypermutators making up half of the initial population.
With 256 agents per deme, population size scaled up to 136 million agents.
In line with expectations, the \textit{de novo} treatment substantially extends normomutator viability.
Under \textit{de novo} conditions, normomutators reliably resist hypermutator fixation in very large populations through more than 10 beneficial mutations available.
Under 50/50 conditions, by contrast, hypermutators begin to fix in very large populations past 2 beneficial mutations available.

As can be seen by comparing panels \ref{fig:denovo-5050-conditions:poisson} and \ref{fig:denovo-5050-conditions:site-explicit}, we found outcomes to be generally consistent between both the counter-based model with Poisson-distributed mutations and an alternate ``site-explicit'' model where the probability of discovering a beneficial mutation scales proportional to the number available.

\subsection{Effects of Population Structure}
\label{sec:population-structure}

In a final set of experiments, we sought to establish the influence of spatial structure on fixation outcomes to further assess how robustly natural asexual populations might resist hypermutator fixation in scenarios with limited adaptation potential.

The simulation experiments described above, conducted on WSE hardware, all assumed a global population structure as an interconnected 2D grid of smaller, well-mixed constituent demes.
To explore this dimension, we conducted trials with the default 2D structure along with (1) a treatment arranging demes in a 1D ring and (2) a fully well-mixed treatment with no subpopulation structure.
Because work reported in this section used a GPU platform, rather than WSE, population sizes ranged only up to 15 million agents.

\begin{figure}[h]
\begin{minipage}{\textwidth}
  \includegraphics[width=\linewidth]{binder/binder-cupy-traits.ipynb/binder/teeplots/cupy-traits/errorbar=ci+exclude=1D-demes+hue=population-structure+kind=line+palette=dark2+row=population-size+style=initial-conditions+viz=relplot+x=available-beneficial-mutations+y=fix-prob+ext=.pdf}%
\end{minipage}

\begin{minipage}{\textwidth}
  \caption{%
    \textbf{Population structure increases normomutator resilience to available adaptive potential.}
    \footnotesize
    Lineplot strips show relationship between adaptive potential and hypermutator fixation probability across surveyed population sizes, with error bands indicating bootstrapped 95\% confidence intervals.
    In the 50/50 treatment, experiments were initialized with an even mix of hyper- and normomutator agents.
    In the \textit{de novo} treatment, hypermutators are initially absent and arise spontaneously \textit{de novo} with probability $10^{-6}$.
    Effects of population structure are more pronounced under \textit{de novo} conditions, when initial supply of hypermutators is scarce.
    Under well-mixed conditions, increase in population size favors hypermutators; normomutators reliably persist in very large populations through fewer than 10 available beneficial mutations.
    In contrast, under 2D deme structure, very large population sizes are favorable to normomutators.
    With such strong spatial structure, very large populations of normomutators resist hypermutator fixation even when upwards of 10 beneficial mutations are available.
    Results are under GPU simulation, using counter-based genome model.
    Supplementary Figures \cref{fig:fixheat-5050-cupy,fig:fixheat-denovo-cupy} detail results in a tabular format.
  }
  \label{fig:spatial-structure-combined}
\end{minipage}
\end{figure}


Figure \ref{fig:spatial-structure} plots increase in hypermutator fixation probabilities with available adaptive potential across a spectrum of population sizes.
Results differ substantially between 50/50 and \textit{de novo} conditions, appearing in panels \ref{fig:spatial-structure:5050} and \ref{fig:spatial-structure:denovo} respectively.
With 50/50 initializations, normomutators reliably persist through only one or two beneficial mutations available across all spatial configurations.
By contrast, under \textit{de novo} conditions, normomutator populations with 1D structure successfully resist hypermutator invasion through upwards of 20 beneficial mutations.
Notably, though, under weak, well-mixed spatial structure, the effect of population size is opposite --- rather than suppressing hypermutator favorability, hypermutators actually become \textit{more} likely to sweep large populations.

These findings contrast with previous modeling work assuming unlimited adaptive potential, which predicts that so long as any level of connectivity exists between subpopulations, spatial structure boosts hypermutator fixation probability \citep{raynes2019migration}.
One possible explanation for why spatial structure can decrease hypermutator fixation probability is by extending the window of opportunity for normomutators to catch up in exploiting available adaptive potential before being driven to extinction.
Such a possibility aligns with findings from \textit{in vivo} evolution experiments with \textit{E. coli} that show that migration barriers delaying introduction of hypermutator strains into normomutator populations can reduce hypermutator advantage \citep{lechat2006escherichia}.

\subsection{Spatiotemporal Dynamics}
\label{sec:dynamics}

To better understand how competition between normo- and hypermutator strains unfolds at very large population scales, we performed an additional set of simulations on the WSE using the site-explicit model of adaptation, where hypermutators originated \textit{de novo} from normomutator strains.
For these experiments, PE subpopulations were restricted to 256 agents to preserve on-device memory to record time series data.
This configuration yielded a net population size of 191 million agents.
In these runs, we probed the critical region where outcomes transition between normomutator and hypermutator fixation, configuring adaptive potential between 12 and 16 available beneficial mutations.

In order to track the composition of the population over time, we included additional on-device, runtime instrumentation in these replicates.
Generation-on-generation, we configured each processor element to sample a genome from its local population and record its hypermutator allele (i.e., a binary value as either a normo- or hypermutator).
To efficiently curate this single-bit data within per-PE memory limits, we leveraged recently-introduced ``DStream'' generalizations of the ring buffer data structure \citep{moreno2024algorithms}.
Notably, this approach constrained memory use to a fixed-size buffer, while guaranteeing the buffer to always be fully utilized.
Additionally, recording could thereby be dynamically halted when normomutators were determined to be locally extinct (and restarted if they are reintroduced).
While this scheme allowed maximization of on-device resources, note that, as a result, in animations the number of generations elapsed per frame varies depending on when, and if, normomutators went to extinction.

\begin{figure}[ht]
  \centering
  %----------------------------
  % First row: NO FIXATION
  \begin{subfigure}[t]{\textwidth}
    \centering
    % Left minipage for row label (empty caption)
    \begin{minipage}[b]{0.04\textwidth}
      \caption{}
      \label{fig:dynamics:no-fixation}
    \end{minipage}
    % Six images (columns)
    \begin{minipage}[b]{0.15\textwidth}
      \includegraphics[width=\linewidth]{binder/binder-wse-denovo-spatial2d-explicitsite-timeseries.ipynb/binder/wse-denovo-spatial2d-explicitsite-timeseries/a=traitframes+nmut=14+rep=39a89ca6-a1b5-4b32-ae5f-f0dbb40ba027/dstream_Tbar=000504+ext=.png}
    \end{minipage}
    \begin{minipage}[b]{0.15\textwidth}
      \includegraphics[width=\linewidth]{binder/binder-wse-denovo-spatial2d-explicitsite-timeseries.ipynb/binder/wse-denovo-spatial2d-explicitsite-timeseries/a=traitframes+nmut=14+rep=5dc8e084-0382-4d7b-9b76-6c3902ca3c1d/dstream_Tbar=000952+ext=.png}
    \end{minipage}
    \begin{minipage}[b]{0.15\textwidth}
      \includegraphics[width=\linewidth]{binder/binder-wse-denovo-spatial2d-explicitsite-timeseries.ipynb/binder/wse-denovo-spatial2d-explicitsite-timeseries/a=traitframes+nmut=14+rep=5dc8e084-0382-4d7b-9b76-6c3902ca3c1d/dstream_Tbar=001400+ext=.png}
    \end{minipage}
    \begin{minipage}[b]{0.15\textwidth}
      \includegraphics[width=\linewidth]{binder/binder-wse-denovo-spatial2d-explicitsite-timeseries.ipynb/binder/wse-denovo-spatial2d-explicitsite-timeseries/a=traitframes+nmut=14+rep=5dc8e084-0382-4d7b-9b76-6c3902ca3c1d/dstream_Tbar=002552+ext=.png}
    \end{minipage}
    \begin{minipage}[b]{0.15\textwidth}
      \includegraphics[width=\linewidth]{binder/binder-wse-denovo-spatial2d-explicitsite-timeseries.ipynb/binder/wse-denovo-spatial2d-explicitsite-timeseries/a=traitframes+nmut=14+rep=5dc8e084-0382-4d7b-9b76-6c3902ca3c1d/dstream_Tbar=020472+ext=.png}
    \end{minipage}
    \begin{minipage}[b]{0.15\textwidth}
      \includegraphics[width=\linewidth]{binder/binder-wse-denovo-spatial2d-explicitsite-timeseries.ipynb/binder/wse-denovo-spatial2d-explicitsite-timeseries/a=traitframes+nmut=14+rep=5dc8e084-0382-4d7b-9b76-6c3902ca3c1d/dstream_Tbar=061424+ext=.png}
    \end{minipage}
  \end{subfigure}

  \vspace{1em} % vertical space between rows

  %----------------------------
  % Second row: FIXATION
  \begin{subfigure}[t]{\textwidth}
    \centering
    % Left minipage for row label (empty caption)
    \begin{minipage}[b]{0.05\textwidth}
      \caption{}
      \label{fig:dynamics:fixation}
    \end{minipage}
    % Six images (columns)
    \begin{minipage}[b]{0.15\textwidth}
      \includegraphics[width=\linewidth]{binder/binder-wse-denovo-spatial2d-explicitsite-timeseries.ipynb/binder/wse-denovo-spatial2d-explicitsite-timeseries/a=traitframes+nmut=14+rep=39a89ca6-a1b5-4b32-ae5f-f0dbb40ba027/dstream_Tbar=000504+ext=.png}
    \end{minipage}
    \begin{minipage}[b]{0.15\textwidth}
      \includegraphics[width=\linewidth]{binder/binder-wse-denovo-spatial2d-explicitsite-timeseries.ipynb/binder/wse-denovo-spatial2d-explicitsite-timeseries/a=traitframes+nmut=14+rep=39a89ca6-a1b5-4b32-ae5f-f0dbb40ba027/dstream_Tbar=000952+ext=.png}
    \end{minipage}
    \begin{minipage}[b]{0.15\textwidth}
      \includegraphics[width=\linewidth]{binder/binder-wse-denovo-spatial2d-explicitsite-timeseries.ipynb/binder/wse-denovo-spatial2d-explicitsite-timeseries/a=traitframes+nmut=14+rep=39a89ca6-a1b5-4b32-ae5f-f0dbb40ba027/dstream_Tbar=001400+ext=.png}
    \end{minipage}
    \begin{minipage}[b]{0.15\textwidth}
      \includegraphics[width=\linewidth]{binder/binder-wse-denovo-spatial2d-explicitsite-timeseries.ipynb/binder/wse-denovo-spatial2d-explicitsite-timeseries/a=traitframes+nmut=14+rep=39a89ca6-a1b5-4b32-ae5f-f0dbb40ba027/dstream_Tbar=002489+ext=.png}
    \end{minipage}
    \begin{minipage}[b]{0.15\textwidth}
      \includegraphics[width=\linewidth]{binder/binder-wse-denovo-spatial2d-explicitsite-timeseries.ipynb/binder/wse-denovo-spatial2d-explicitsite-timeseries/a=traitframes+nmut=14+rep=39a89ca6-a1b5-4b32-ae5f-f0dbb40ba027/dstream_Tbar=002489+ext=.png}
    \end{minipage}
    \begin{minipage}[b]{0.15\textwidth}
      \includegraphics[width=\linewidth]{binder/binder-wse-denovo-spatial2d-explicitsite-timeseries.ipynb/binder/wse-denovo-spatial2d-explicitsite-timeseries/a=traitframes+nmut=14+rep=39a89ca6-a1b5-4b32-ae5f-f0dbb40ba027/dstream_Tbar=002489+ext=.png}
    \end{minipage}
  \end{subfigure}

  %----------------------------
  % Timepoint labels below the bottom row:
  \begin{minipage}[c]{0.154\textwidth}
\hfill
\begin{varwidth}{\textwidth}
$T = 504$
\end{varwidth}
\hfill
  \end{minipage}
  \begin{minipage}[c]{0.154\textwidth}
\hfill
\begin{varwidth}{\textwidth}
$T = 952$
\end{varwidth}
\hfill
  \end{minipage}
  \begin{minipage}[c]{0.154\textwidth}
\hfill
\begin{varwidth}{\textwidth}
$T = 1,400$
\end{varwidth}
\hfill
  \end{minipage}
  \begin{minipage}[c]{0.154\textwidth}
\hfill
\begin{varwidth}{\textwidth}
$T = 2,552$
\end{varwidth}
\hfill
  \end{minipage}
  \begin{minipage}[c]{0.154\textwidth}
\hfill
\begin{varwidth}{\textwidth}
$T = 20,472$
\end{varwidth}
\hfill
  \end{minipage}
  \begin{minipage}[c]{0.154\textwidth}
\hfill
\begin{varwidth}{\textwidth}
$T = 61,424$
\end{varwidth}
\hfill
  \end{minipage}
  \caption{
  \textbf{Spatiotemporal composition of simulated populations from Wafer-Scale Engine experiments.}
  \footnotesize
  Snapshots show 191 million agent populations with 2D spatial structure, using site-explicit genome model configured to adaptive potential of 14 beneficial mutations available and hypermutators introduced \textit{de novo}.
  Raster values are binary, with white pixels indicating a sampled normomutators and black pixels indicating a sampled hypermutator.
  Subpanels \ref{fig:dynamics:no-fixation} and       \ref{fig:dynamics:fixation} show replicates where hypermutators do not, and do, reach fixation, respectively.
  Animations are provided at \url{https://hopth.ru/ej} and \url{https://hopth.ru/ek}.
  Example timecourses for simulations with 12 and 16 beneficial mutations available can be found at \url{https://hopth.ru/el} and \url{https://hopth.ru/em}.
  }
  \label{fig:dynamics}
\end{figure}


Figure \ref{fig:dynamics} arranges sequential population snapshots from two example simulations, both with adaptive potential of 14 beneficial mutations.
In the first example, hypermutators fixed; in the second, they did not.

At outset, example simulations in panels \ref{fig:dynamics:no-fixation} and \ref{fig:dynamics:fixation} unfold similarly.
First, hypermutator strains appear \textit{de novo} across the breadth of simulated populations.
Then, as patches of hypermutators expand outwards, the normomutator population is broken into pockets.%
\footnote{%
Notably, hypermutator strains continue to stem from normomutator lineages as they accrue adaptive mutations.
Of 16 replicates where hypermutators fixed, we observed 2 cases where the final dominant hypermutator lineage arose from a partially-adapted normomutator lineage --- harboring 1 and 3 beneficial mutations, respectively.
% https://github.com/mmore500/hypermutator-dynamics/blob/90422b30f2dd9ca29baad273cbcc676dbfeab55f/binder/wse-denovo-spatial2d-explicitsite-genomes.ipynb
}
In scenarios where hypermutators fix, surviving patches of normomutators dwindle away entirely (panel \ref{fig:dynamics:fixation}).
Alternately, where normomutators survive, they re-establish via concentric growth from surviving pockets (panel \ref{fig:dynamics:no-fixation}).

\begin{figure}

\begin{minipage}{0.4\textwidth}
\includegraphics[width=\textwidth]{binder/binder-wse-denovo-spatial2d-explicitsite-timeseries-2025-02-07-peaksweep.ipynb/binder/teeplots/wse-denovo-spatial2d-explicitsite-timeseries-2025-02-07-peaksweep/color=orange+kind=line+viz=relplot+x=cerebraslib-hypermut-num-avail-ben-muts+y=dstream-value+ext=.pdf}
\end{minipage}
\begin{minipage}{0.55\textwidth}
\caption{
\textbf{Mutators prevalence transitively peaks above population majority in scenarios with limited adaptive potential.}
\footnotesize
Simulations were conducted on WSE, configured with population size 191 million over 2D spatial structure and site-explicit genome model with mutators introduced \textit{de novo}.
Under these conditions, mutator fixation was observed only for 14 or more available beneficial mutations.
Shaded bands show 100\% percentile interval across four replicates.
}
\label{fig:peaksweep}
\end{minipage}
\end{figure}


At nadir, normomutators constitute less than 0.1\% of the population in experiments with adaptive potential of 14 available beneficial mutations.
% ^^^ https://github.com/mmore500/hypermutator-dynamics/blob/5e904ac2fbbd4b41a0ad679883fc1a63af71c00c/binder/wse-denovo-spatial2d-explicitsite-timeseries.ipynb
To assess the peak of transient presence of hypermutators within, we performed an additional set of four-replicate simulations ranging from adaptive potentials of 2 to 10 available beneficial mutations.
For all replicates with 6 or more available beneficial mutations, hypermutators peaked beyond 50\% prevalence (Figure \ref{fig:peaksweep}).
